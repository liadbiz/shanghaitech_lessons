\documentclass{article}

\usepackage{fancyhdr}
\usepackage{extramarks}
\usepackage{amsmath}
\usepackage{amsthm}
\usepackage{amsfonts}
\usepackage{tikz}
\usepackage[plain]{algorithm}
\usepackage{algpseudocode}
\usepackage{enumerate}

\usetikzlibrary{automata,positioning}

%
% Basic Document Settings
%  

\topmargin=-0.45in
\evensidemargin=0in
\oddsidemargin=0in
\textwidth=6.5in
\textheight=9.0in
\headsep=0.25in

\linespread{1.1}

\pagestyle{fancy}
\lhead{\hmwkAuthorName}
\chead{\hmwkClass\ (\hmwkClassInstructor\ \hmwkClassTime): \hmwkTitle}
\rhead{\firstxmark}
\lfoot{\lastxmark}
\cfoot{\thepage}

\renewcommand\headrulewidth{0.4pt}
\renewcommand\footrulewidth{0.4pt}

\setlength\parindent{0pt}

%
% Create Problem Sections
%

\newcommand{\enterProblemHeader}[1]{
    \nobreak\extramarks{}{Problem \arabic{#1} continued on next page\ldots}\nobreak{}
    \nobreak\extramarks{Problem \arabic{#1} (continued)}{Problem \arabic{#1} continued on next page\ldots}\nobreak{}
}

\newcommand{\exitProblemHeader}[1]{
    \nobreak\extramarks{Problem \arabic{#1} (continued)}{Problem \arabic{#1} continued on next page\ldots}\nobreak{}
    \stepcounter{#1}
    \nobreak\extramarks{Problem \arabic{#1}}{}\nobreak{}
}

\setcounter{secnumdepth}{0}
\newcounter{partCounter}
\newcounter{homeworkProblemCounter}
\setcounter{homeworkProblemCounter}{1}
\nobreak\extramarks{Problem \arabic{homeworkProblemCounter}}{}\nobreak{}

%
% Homework Problem Environment
%
% This environment takes an optional argument. When given, it will adjust the
% problem counter. This is useful for when the problems given for your
% assignment aren't sequential. See the last 3 problems of this template for an
% example.
%
\newenvironment{homeworkProblem}[1][-1]{
    \ifnum#1>0
        \setcounter{homeworkProblemCounter}{#1}
    \fi
    \section{Problem \arabic{homeworkProblemCounter}}
    \setcounter{partCounter}{1}
    \enterProblemHeader{homeworkProblemCounter}
}{
    \exitProblemHeader{homeworkProblemCounter}
}

%
% Homework Details
%   - Title
%   - Due date
%   - Class
%   - Section/Time
%   - Instructor
%   - Author
%

\newcommand{\hmwkTitle}{Homework\ \#1}
\newcommand{\hmwkDueDate}{October 10, 2017}
\newcommand{\hmwkClass}{Stochastic Processes}
\newcommand{\hmwkClassTime}{Lecture 1}
\newcommand{\hmwkClassInstructor}{Professor Ziyu Shao}
\newcommand{\hmwkAuthorName}{zhuhaihao}

%
% Title Page
%

\title{
    \vspace{2in}
    \textmd{\textbf{\hmwkClass:\ \hmwkTitle}}\\
    \normalsize\vspace{0.1in}\small{Due\ on\ \hmwkDueDate\ at 9:00am}\\
    \vspace{0.1in}\large{\textit{\hmwkClassInstructor\ \hmwkClassTime}}
    \vspace{3in}
}

\author{\textbf{\hmwkAuthorName}}
\date{}

\renewcommand{\part}[1]{\textbf{\large Part \Alph{partCounter}}\stepcounter{partCounter}\\}

%
% Various Helper Commands
%

% Useful for algorithms
\newcommand{\alg}[1]{\textsc{\bfseries \footnotesize #1}}

% For derivatives
\newcommand{\deriv}[1]{\frac{\mathrm{d}}{\mathrm{d}x} (#1)}

% For partial derivatives
\newcommand{\pderiv}[2]{\frac{\partial}{\partial #1} (#2)}

% Integral dx
\newcommand{\dx}{\mathrm{d}x}

% Alias for the Solution section header
\newcommand{\solution}{\textbf{\large Solution}}

% Probability commands: Expectation, Variance, Covariance, Bias
\newcommand{\E}{\mathrm{E}}
\newcommand{\Var}{\mathrm{Var}}
\newcommand{\Cov}{\mathrm{Cov}}
\newcommand{\Bias}{\mathrm{Bias}}

\begin{document}

\maketitle

\pagebreak

\begin{homeworkProblem}
	A round-robin tournament is being held with n tennis players; this means that every
player will play against every other player exactly once.
	\\
	(a) How many possible outcomes are there for the tournament (the outcome lists out
who won and who lost for each game)?
	\\
	(b) How many games are played in total?
	\\
	\\
    \textbf{Solution}
	\\
    \textbf{Question One}
	\\
	First we have to known how many games are played in total. according to what round-robin tournament means, we known that we have:
	\[
		\binom{n}{2} = \frac{n \cdot (n - 1)}{2}
	\]
    and for each game, there are 2 possible outcomes, so we get \(n \cdot (n - 1)\) possible outcomes in total.
    \\

    \textbf{Question Two}
    \\
	As is claimed in Question one, There are \(\frac{n \cdot (n - 1)}{2}\) played in total.

\end{homeworkProblem}

\begin{homeworkProblem}
    A certain casino uses 10 standard decks of cards mixed together into one big deck, which
we will call a superdeck. Thus, the superdeck has 52 · 10 = 520 cards, with 10 copies
of each card. How many different 10-card hands can be dealt from the superdeck? The
order of the cards does not matter, nor does it matter which of the original 10 decks
the cards came from. Express your answer as a binomial coefficient.
Hint: Bose-Einstein.
    \\
	\\
    \textbf{Solution}
	\\
    
	Since we choose ten cards from 520 cards, and The order of the cards does not matter, nor does it matter which of the original 10 decks the cards came from, for each kinds of combination for these ten cards, we only need to consider how many times one type of card appear. Thus we can use Bose-Einstein to solve this problem.
	\\
	suppose \(x_1, x_2, \ldots, x_52\) is times of each card appear, we can easily get:
	\\
	\[
		\sum_{k=1}^{5} x_k = 10
	\] 
    and then we let \(y_k = x_k + 1\) for \(k = 1, 2, \ldots, 52\), so can use formula of Bose-Einstein Counting  problem. We can get \(\binom{61}{51}\) different 10-card from the superdeck.

\end{homeworkProblem}

\begin{homeworkProblem}
    Write part of \alg{Quick-Sort($list, start, end$)}

    \begin{algorithm}[]
        \begin{algorithmic}[1]
            \Function{Quick-Sort}{$list, start, end$}
                \If{$start \geq end$}
                    \State{} \Return{}
                \EndIf{}
                \State{} $mid \gets \Call{Partition}{list, start, end}$
                \State{} \Call{Quick-Sort}{$list, start, mid - 1$}
                \State{} \Call{Quick-Sort}{$list, mid + 1, end$}
            \EndFunction{}
        \end{algorithmic}
        \caption{Start of QuickSort}
    \end{algorithm}
\end{homeworkProblem}

\pagebreak

\begin{homeworkProblem}
    Suppose we would like to fit a straight line through the origin, i.e.,
    \(Y_i = \beta_1 x_i + e_i\) with \(i = 1, \ldots, n\), \(\E [e_i] = 0\),
    and \(\Var [e_i] = \sigma^2_e\) and \(\Cov[e_i, e_j] = 0, \forall i \neq
    j\).
    \\

    \part

    Find the least squares esimator for \(\hat{\beta_1}\) for the slope
    \(\beta_1\).
    \\

    \solution

    To find the least squares estimator, we should minimize our Residual Sum
    of Squares, RSS:

    \[
        \begin{split}
            RSS &= \sum_{i = 1}^{n} {(Y_i - \hat{Y_i})}^2
            \\
            &= \sum_{i = 1}^{n} {(Y_i - \hat{\beta_1} x_i)}^2
        \end{split}
    \]

    By taking the partial derivative in respect to \(\hat{\beta_1}\), we get:

    \[
        \pderiv{
            \hat{\beta_1}
        }{RSS}
        = -2 \sum_{i = 1}^{n} {x_i (Y_i - \hat{\beta_1} x_i)}
        = 0
    \]

    This gives us:

    \[
        \begin{split}
            \sum_{i = 1}^{n} {x_i (Y_i - \hat{\beta_1} x_i)}
            &= \sum_{i = 1}^{n} {x_i Y_i} - \sum_{i = 1}^{n} \hat{\beta_1} x_i^2
            \\
            &= \sum_{i = 1}^{n} {x_i Y_i} - \hat{\beta_1}\sum_{i = 1}^{n} x_i^2
        \end{split}
    \]

    Solving for \(\hat{\beta_1}\) gives the final estimator for \(\beta_1\):

    \[
        \begin{split}
            \hat{\beta_1}
            &= \frac{
                \sum {x_i Y_i}
            }{
                \sum x_i^2
            }
        \end{split}
    \]

    \pagebreak

    \part

    Calculate the bias and the variance for the estimated slope
    \(\hat{\beta_1}\).
    \\

    \solution

    For the bias, we need to calculate the expected value
    \(\E[\hat{\beta_1}]\):

    \[
        \begin{split}
            \E[\hat{\beta_1}]
            &= \E \left[ \frac{
                \sum {x_i Y_i}
            }{
                \sum x_i^2
            }\right]
            \\
            &= \frac{
                \sum {x_i \E[Y_i]}
            }{
                \sum x_i^2
            }
            \\
            &= \frac{
                \sum {x_i (\beta_1 x_i)}
            }{
                \sum x_i^2
            }
            \\
            &= \frac{
                \sum {x_i^2 \beta_1}
            }{
                \sum x_i^2
            }
            \\
            &= \beta_1 \frac{
                \sum {x_i^2 \beta_1}
            }{
                \sum x_i^2
            }
            \\
            &= \beta_1
        \end{split}
    \]

    Thus since our estimator's expected value is \(\beta_1\), we can conclude
    that the bias of our estimator is 0.
    \\

    For the variance:

    \[
        \begin{split}
            \Var[\hat{\beta_1}]
            &= \Var \left[ \frac{
                \sum {x_i Y_i}
            }{
                \sum x_i^2
            }\right]
            \\
            &=
            \frac{
                \sum {x_i^2}
            }{
                \sum x_i^2 \sum x_i^2
            } \Var[Y_i]
            \\
            &=
            \frac{
                \sum {x_i^2}
            }{
                \sum x_i^2 \sum x_i^2
            } \Var[Y_i]
            \\
            &=
            \frac{
                1
            }{
                \sum x_i^2
            } \Var[Y_i]
            \\
            &=
            \frac{
                1
            }{
                \sum x_i^2
            } \sigma^2
            \\
            &=
            \frac{
                \sigma^2
            }{
                \sum x_i^2
            }
        \end{split}
    \]

\end{homeworkProblem}

\pagebreak

\begin{homeworkProblem}
    Prove a polynomial of degree \(k\), \(a_kn^k + a_{k - 1}n^{k - 1} + \hdots
    + a_1n^1 + a_0n^0\) is a member of \(\Theta(n^k)\) where \(a_k \hdots a_0\)
    are nonnegative constants.

    \begin{proof}
        To prove that \(a_kn^k + a_{k - 1}n^{k - 1} + \hdots + a_1n^1 +
        a_0n^0\), we must show the following:

        \[
            \exists c_1 \exists c_2 \forall n \geq n_0,\ {c_1 \cdot g(n) \leq
            f(n) \leq c_2 \cdot g(n)}
        \]

        For the first inequality, it is easy to see that it holds because no
        matter what the constants are, \(n^k \leq a_kn^k + a_{k - 1}n^{k - 1} +
        \hdots + a_1n^1 + a_0n^0\) even if \(c_1 = 1\) and \(n_0 = 1\).  This
        is because \(n^k \leq c_1 \cdot a_kn^k\) for any nonnegative constant,
        \(c_1\) and \(a_k\).
        \\

        Taking the second inequality, we prove it in the following way.
        By summation, \(\sum\limits_{i=0}^k a_i\) will give us a new constant,
        \(A\). By taking this value of \(A\), we can then do the following:

        \[
            \begin{split}
                a_kn^k + a_{k - 1}n^{k - 1} + \hdots + a_1n^1 + a_0n^0 &=
                \\
                &\leq (a_k + a_{k - 1} \hdots a_1 + a_0) \cdot n^k
                \\
                &= A \cdot n^k
                \\
                &\leq c_2 \cdot n^k
            \end{split}
        \]

        where \(n_0 = 1\) and \(c_2 = A\). \(c_2\) is just a constant. Thus the
        proof is complete.
    \end{proof}
\end{homeworkProblem}

\pagebreak

%
% Non sequential homework problems
%

% Jump to problem 18
\begin{homeworkProblem}[18]
    Evaluate \(\sum_{k=1}^{5} k^2\) and \(\sum_{k=1}^{5} (k - 1)^2\).
\end{homeworkProblem}

% Continue counting to 19
\begin{homeworkProblem}
    Find the derivative of \(f(x) = x^4 + 3x^2 - 2\)
\end{homeworkProblem}

% Go back to where we left off
\begin{homeworkProblem}[6]
    Evaluate the integrals
    \(\int_0^1 (1 - x^2) \dx\)
    and
    \(\int_1^{\infty} \frac{1}{x^2} \dx\).
\end{homeworkProblem}

\end{document}
